%!TeX root = main-systeme.tex


% --- PRÉAMBULE  --- 

% langue et police:
\usepackage[quiet]{fontspec}
\usepackage{polyglossia} \setmainlanguage[variant=swiss]{french}

% calculs internes:
\usepackage{luacas}
%\usepackage[pl,import]{penlight} % désuet

% maths:
\usepackage{amsmath} 
\usepackage{amssymb}
\usepackage[ % choix esthétique, facultatif
warnings-off={mathtools-colon,mathtools-overbracket},
math-style=ISO,
]{unicode-math} 
\usepackage[ % à régler selon préférences / change as preferred
% utilisé ici pour formatter les nombres à virgule avec \num{}
% used here to format decimal numbers via \num{}
locale = FR,
round-precision = 3,
round-mode = figures,
round-pad = false,
%zero-decimal-as-symbol = true, % choix personnel / personal choice
]{siunitx} 
% vvvvvvvvvvvvvvvvvvvvv
%	\newcommand{\num}[1]{#1}
%% 	^ à décommenter si vous voulez vous débarasser de ‹siunitx›
%% 	^ uncomment if you want to get rid of ‹siunitx›

% choix esthétiques, facultatifs:
\let\oldemptyset\emptyset
\let\emptyset\varnothing

% permettent mise en page:
\usepackage{multicol}
\usepackage{pgffor}

% date et heure
\usepackage[ % permet les horodatages (timestamps)
timesep=., 
showzone=false, 
hourminsep=h, 
minsecsep=m
]{datetime2}


% ---- ANSWER-PRINTING STYLE FROM https://tex.stackexchange.com/a/15354 ----

% Define answer environment 

\newbox\allanswers
\setbox\allanswers=\vbox{}

\newenvironment{customanswer}
{\global\setbox\allanswers=\vbox\bgroup
	\unvbox\allanswers
	\vspace{-4pt}
}
{\bigbreak\egroup}

\newcommand{\showallanswers}{\par\unvbox\allanswers}

% Define question environment

\newbox\allquestions
\setbox\allquestions=\vbox{}

\newenvironment{customquestion}
{\global\setbox\allquestions=\vbox\bgroup
	\unvbox\allquestions
}
{\bigbreak\egroup}

\newcommand{\showallquestions}{\par\unvbox\allquestions}

% --------------------------------------------------------------------------



%% ---------- PROVIDE \timestamp COMMAND -------------
% -- from https://flaterco.com/util/timestamp.sty --

\makeatletter

\newcount\@DT@modctr
\newcount\@dtctr

\def\@modulo#1#2{%
	\@DT@modctr=#1\relax
	\divide \@DT@modctr by #2\relax
	\multiply \@DT@modctr by #2\relax
	\advance #1 by -\@DT@modctr}

\newcommand{\xxivtime}{%
	\@dtctr=\time%
	\divide\@dtctr by 60
	\ifnum\@dtctr<10 0\fi\the\@dtctr.%
	\@dtctr=\time%
	\@modulo{\@dtctr}{60}%
	\ifnum\@dtctr<10 0\fi\the\@dtctr%
}

\newcommand{\timestamp}{\the\year-%
	\ifnum\month<10 0\fi\the\month-%
	\ifnum\day<10 0\fi\the\day\ \xxivtime}

\makeatother

% -----------------------------------------------------



% ---------------- ACCÈS AU CODE LUA: -----------------

% Aide-mémoire Lua: https://devhints.io/lua

% Activer les fonctions dans le code:
% Load lua code:
\directlua{codeB = require "codeSysteme"} 

% Convertir fonctions de code.lua en commandes LaTeX:
% Extract useful functions as LaTeX commands:

% Pour le format par défaut / for default style:
\newcommand{\bfullroutine}[1]{\directlua{codeB.fullRoutine(#1)}}

% Pour imprimer question immédiatement suivie de réponse:
% To print question immediately followed by answer:
\newcommand{\bprintqna}{\directlua{codeB.printQnA()}}
